\documentclass[12 pt]{exam}
\usepackage{graphicx, enumitem, amsmath, amssymb}
\graphicspath{ {./images/} }
\usepackage{tikz, pgfplots}
\usetikzlibrary{shapes,arrows}
%\usepackage{Minion Pro}
\printanswers

\title{1.7 Building a Demand Function - Practice Problems (Answers)}
\author{Ryan Safner}
\date{ECON 306 - Fall 2019}

\begin{document}

\maketitle

You can spend your income on apples and oranges. Apples currently cost \$0.25 and oranges cost \$0.50. When your income is \$40, you buy 10 apples and 8 oranges. When your income increases to \$80, you buy 12 apples and 6 oranges

\begin{questions}

\question What type of good are apples (inferior, necessity, luxury)?

\begin{solution}

\begin{equation*}
\displaystyle \cfrac{\left(\frac{a_2-a_1}{a_1}\right)}{\left(\frac{m_2-m_1}{m_1}\right)} = \cfrac{\left(\frac{12-10}{10}\right)}{\left(\frac{80-40}{40}\right)}=\cfrac{\left(\frac{2}{10}\right)}{\left(\frac{40}{40}\right)} = \frac{0.20}{1} = 0.20
\end{equation*}

Apples are (normal) necessity goods. For every 1\% increase in income, you buy 0.20\% more apples.

\end{solution}

\question What type of good are oranges (inferior, necessity, or luxury)?

\begin{solution}

\begin{equation*}
\displaystyle \cfrac{\left(\frac{o_2-o_1}{o_1}\right)}{\left(\frac{m_2-m_1}{m_1}\right)} = \cfrac{\left(\frac{6-8}{8}\right)}{\left(\frac{80-40}{40}\right)}=\cfrac{\left(\frac{-2}{8}\right)}{\left(\frac{40}{40}\right)} = \frac{-0.25}{1} = -0.25
\end{equation*}

Apples are inferior goods. For every 1\% increase in income, you buy 0.25\% \emph{fewer} oranges.

\end{solution}

\clearpage

You can can have cereal and milk for breakfast. When milk is \$2/gallon, you consume 5 bowls of cereal per week. When milk increases to \$4/gallon, you consume 4 bowls of cereal per week.

\question What is the relationship between these two goods?

\begin{solution}
These goods are complements.
\end{solution}

\question What is the cross-price elasticity? 

\begin{solution}

\begin{equation*}
\displaystyle \cfrac{\left(\frac{c_2-c_1}{c_1}\right)}{\left(\frac{pm_2-pm_1}{pm_1}\right)} = \cfrac{\left(\frac{4-5}{5}\right)}{\left(\frac{4-2}{2}\right)}=\cfrac{\left(\frac{-1}{5}\right)}{\left(\frac{2}{2}\right)} = \frac{-0.20}{1} = -0.20
\end{equation*}

For every 1\% increase in the price of milk, you buy 0.20\% \emph{fewer} boxes of cereal.

\end{solution}

\end{questions}

\end{document}